\documentclass[10pt,a4paper,oneside]{book}
\usepackage[letterpaper, left=2cm, top=2cm, right=2cm, bottom=2cm]{geometry}
\usepackage[spanish]{babel}
\usepackage[utf8]{inputenc}
\usepackage{amsmath}
\usepackage{amsfonts}
\usepackage{amssymb}
\usepackage{graphicx}
\usepackage{listings}

\begin{document}

\section*{Instalacion}
\begin{enumerate}
\item yaourt -Syua geant4 geant4-abladata geant4-ensdfstatedata geant4-ledata
  geant4-levelgammadata geant4-neutronhpdata geant4-neutronxsdata
  geant4-piidata geant4-radioactivedata geant4-realsurfacedata
  geant4-saiddata
\end{enumerate}


\section*{CMake}
\begin{enumerate}
\item Este man escribe los makefile
\item Hay que ejecutarlo nuevamente cuando se cambie la jerarquia de
  directorios.
\item CMake soporta la generación de ficheros para varios sistemas
  operativos
\item CMake es para Make como el compilador de C/C++ para el lenguaje
  ensamblador.
\end{enumerate}

\section*{C/C++}
Ahora hablemos de algunos conceptos fundamentales involucrados en
programacion, por el momento particularicemolo a c++. Aunque estas
definiciones pueden hacerse generales en otros lenguajes.
\begin{enumerate}
\item Principio de privilegio minimo: el codigo debe dar prinvilegio y
  acceso unicamente para completar una tarea especifica y no mas.
\item La clase es la unidad fundamental en c++, mientras que la
  función es la unidad fundamental en c.
\item Una categoria es el tipo de objeto, clase , funcion, casa...
\item Podemos llamar los integrantes de una clase como miembros, si
  los miembros son funciones decimos que son metodos, si los miembros
  son varibles o datos decimos que son data members o atributos. Es
  decir todos los objetos de la clase pueden manipular su atributos
  atraves de los metodos de la clase.
\item Una asociación entre clases, es básicamente una relación. Por
  ejemplo la herencia es un tipo de asociación.
\item Un identificador es el nombre que le ponenos a un tipo de
  variable u objeto.
\item *Storage Class (Tipo de almacenamiento): determina el periodo
  durante el cual un identificado existe en la memoria por ejemplo
  (static y auto) -> Este concepto esta mas relacionado con el tiempo
\item auto: indica que la variable existe mientras el bloque donde se
  creo este activo. Se destruyen cuando el bloque ya no este
  activo. Solo las variables locales pueden ser una clase de
  almacenamiento automatico. La mayorias de variables locales se
  definen por defecto.
\item static y extern: existen desde que se inicia la ejecución hasta
  que el programa termina
\item *Scope (Alcance): hasta donde un identificador puede ser
  referenciado en un programa. -> Este concepto esta mas relacionado
  con el lugar.
\item Tipos de miembros de una clase: public(que lo puedo llamar por
  fuera.), private (No se puede llamar por fuera de la clase,
  unicamente las funciones miembro tienen acceso), protected (Tanta la
  clase base como sus clases derivadas tienen acceso ) en el caso de
  una herencia los miembros los protegidos se heredan como privados.
  
\end{enumerate}

Muchos ejemplos de Geant4 estan escritos con codigo mixto entonces
acontinuacion estan algunas funciones de C usadas usualmente en Geant
\begin{enumerate}
\item $\#$define : permite definir macros (no los puede modificar el
  codigo) el compilador cambia todo lo que tenga definido el macro por
  su valor. Es un preprocesador de C.
\item const : usualmente usado para definir las constantes globales
  del codigo (es una variable que el compilador entiende que no puede
  modificar)
\end{enumerate}

Ahora mencionemos unos cuantos conceptos nuevos importantes
involucrados, que nos podemos encontrar en la implementacion de
geant4.
\begin{enumerate}
\item El concepto de virtual esta asociado a dos aplicaciones. Todas
  directamente relacionados con el concepto de herencia supongamos que
  tenemos una clase y que en una aplicacion solo necesitamos esa
  clase. Definirle metodos virtuales no tendría ningun sentido si esa
  clase no es la clase base para una futura herencia ya que esos
  metodos no estarian definidos nunca. Esta idea general es lo que se
  conocer como polimorfismo. es decir los metodos de las clases son
  heredados por otras clases pero tiene otra "forma" porque se estan
  sobrecargando con otras firmas("los parametros del metodo son
  diferentes")
\item Poner un metodo virtual en la clase implica que para que los
  objetos de la clase sean reales que la clase que hereda estos
  metodos definan todos los miembros y atributos virtuales.
\item the virtual function can be overriden and the pure virtual must
  be implemented.
\item Derived classes frequently, but not always, override the virtual
  functions that they inherit. If a derived class does not override a
  virtual from its base, then, like any other member, the derived
  class inherits the version defined in its base class.
\item El tipo de variable static esta asociado a que todos los objetos
  de la clase tienen definido el mismo miembro o metodo, por eso es
  importante para llamar siempre al mismo atributo o miembro de una
  clase cada ves que lo necesito, por ejemplo el atributo puede ser la
  clase miembro y el metodo el que permite acceder a esa clase
  miembro, de esta forma siempre se accede a la mismo metodo que
  devuelve la clase miembro y a la misma clase miembro. Lo cual puede
  ser util para instanciar siempre el mismo objeto de la clase.
\item Las funciones tambien pueden tener dentro variables estaticas
  que cada vez que se entra a la funcion usan la misma variable no se
  definen nuevamente. Por ejemplo el numero de cuentas de personas que
  nacen o mueren en un juego y que la funcion o metodo sea disparar
  cada ves que se dispare se usara siempre el mismo valor estatico de
  numero de personas que puede ir cambiando. Seria algo asi como una
  variable global que solo puede modificar es funcion.
\item const son variables que permanecen fijas durante la simulación.
  La forma de inicializar los const de una clase base a traves de
  herencia es con () y comas, aunque de esta forma se puede
  inicializar en general cualquier variable. En el constructor de la
  clase.
\item extern es un tipo de variable global que puede ser usada por
  diferentes estruscturas de datos con el fin de cambiar el alcance de
  la variable.
\item typedef permite definir un alias para tipos de variables. Por
  ejemplo: typedef double wages. Significa que wages es sinonimo de
  doubles
\item Un apuntador es una dirección de memoria, por eso todo apuntador
  debe eliminarse despues que se deje de usar la memoria.
\end{enumerate}

\section*{Geant4 Introduccion}
\begin{enumerate}
\item La clase mas importante de todas son runmanager y
  multithreadmanager.  En el fondo de estas clases estan los metodos
  motecarlo. Y esta son las que administra la simulacion. Realmente el
  mutitread manager es una composicion de G4Manager
\item Hay dos tipos de classes de usuario las $\textbf{Action}$ y las
  \textbf{Initialization}. Las inicialization las debe establecer el
  G4RunManager a traves de los métodos SetUserInitialization(),
  mientras que las classes action deberian se definidas por
  G4VUserActionInitialization. Usualmente se crea una clase que era de
  esta para hacer dicho proceso.
\item Hay tres Initialization clases que son obligatorias:
  G4VUserDetectorConstruction, G4VUserPhysicsList y
  G4VUserActionInitialization. EL RunManager verifica la existencia de
  estas clases cuando se llama los metodos Initialize() y
  BeamOn(). G4VUserActionInitialization obligatoriamente debe incluir
  G4VUserPrimaryGeneratorAction.
\item De todas las clases Action la única que es obligatoria de
  definir es G4VUserPrimaryGeneratorAction. Adicionalmente Geant4
  suministra cinco clases Action. Que permiten acceder a los
  diferentes niveles de la simulación.
  \begin{enumerate}
  \item G4UserRunAction
  \item G4UserEventAction
  \item G4UserStackingAction
  \item G4UserTrackingAction
  \item G4UserSteppingAction
  \end{enumerate}

\item G4UImanger permite invocar metodos de clases de objetos en los
  cuales no se sepa el apuntador. Muy útil por ejemplo para ejecutar
  macros:
  \begin{enumerate}
  \item $ UI->ApplyCommand("/run/verbose 1");$
  \item $ UImanager->ApplyCommand("/control/execute init.mac"); $
  \end{enumerate}

\item G4VUserPhysicsList tiene la responsabilidad de definir todas las
  partículas y procesos físicos que se debe usar en la
  simulación. Este man crea objetos G4ProcessManager para todas las
  particulas definidas en el método puramente virtual
  ConstructParticle(). Además en este método se deben incluir TODAS
  las particulas secundarias o primarias que se quieran utilizar en la
  simulación. A traves de el método puramente virtual
  ConstructProcess() se crean los procesos físicos y se registran
  atraves de las instanciaciones de G4ProcessManager de cada partícula.
\item Son 3 las librerias fundamentales para correr cualquier
  simulación en Geant: 1 Detector Construction, 2. Physics List,
  3. Primary Generation Action. La physicslist puede crearse
  manualmente o insertandole una libreria ya creada. El primary
  Generation Action, es la fabrica de particulas que iniciaran la
  accion en la simulacion, esta incializacion puede hacerse
  manualmente con ParticleGun (codigo directo )o usando General
  Particle Source (GPS) el cual permite definir el tipo la energia, la
  posicion, la direccion y el tipo de distribucion, en un macro o via
  lineas de comando.
\end{enumerate}

\section*{Como construir un detector}
\begin{enumerate}
\item World: EL volumen mas grande que hay que definir.
\item Solid Volume: La forma del volumen.
\item Logic Volume: Incluye todas las propiedades del solido, como la
  geometria del solido, y las propiedades fisicas, material del
  volumen, si contiene elementos de detector sensible como campos
  magneticos, etc.
\item Physical Volume: posiciona una copia del volumen logico dentro
  de un volumen "madre", la posicion implica un lugar y una rotación
  en el sistema de coordenadas respecto al origen de la madre. El
  sistema de coordenadas global  lo define por defecto el volumen
  lógico del mundo.
\end{enumerate}

\section*{Como especificar Materiales in el detector}
\begin{enumerate}
\item G4Isotope = Unidad fundamental. (Z, A, mass por mol)
\item G4Element = combinación de isotopos. (Z efectivo, A efectivo,
  masa efectiva, numero de isotopos, sección eficaz por átomo )
\item G4Material = Compuestos y mezclas. (densidad, estado, temperatura, presión, camino libre medio, dE/dx)
\end{enumerate}
\begin{enumerate}
\item Unicamente G4Material es el que se usa en el resto de la simulación, 
\end{enumerate}

\section*{Como especificar Particulas}
\begin{enumerate}
\item Se debe crear una herencia de G4VUserPhysicsList, ademas se
  deben implementar los métodos ConstructParticle(),
  ConstructProcess() y seguramente se deba sobreescribir el método
  virtual SetCuts(). El métodos ConstructParticle() es puramente
  virtual. Aca se deben incluir TODAS las particulas secundarias o
  primarias que se quieran utilizar en la simulación
\item Todas las partículas son herencia de G4ParticleDefinition
  exceptuando G4Ions. Todas las particulas nuevas que se quieran
  crear, basicamente deben ser herencia de seis categorías mayores.
  \begin{enumerate}
  \item lepton
  \item meson
  \item baryon
  \item boson
  \item shortlived
  \item ion
  \end{enumerate}

\item
  \begin{lstlisting}[language=C++, frame = single]
    void MyPhysicsList::ConstructParticle()
    {
      G4Proton::ProtonDefinition();
      G4Geantino::GeantinoDefinition();
    }
  \end{lstlisting}

  
\item
  \begin{lstlisting}[language=C++, frame = single]
   void ExN05PhysicsList::ConstructLeptons()
   {
     G4LeptonConstructor pConstructor;
     pConstructor.ConstructParticle();
   }
 \end{lstlisting}

\item Con SetCuts se evita que las particulas recorran mas de lo que
  pueda llegar a ser interes, Es util no solo para ahorar computo sino
  tambien para evitar fenomenos fisicos como la divergencia del
  infrarojo. Los cuts se definen en distancia y se convierten
  internamente a energia dependiendo de la particula y del material.
  Por defecto en el limite superior en energía es 10 GeV y el limite
  inferior es 1 mm para todas las partículas.
\end{enumerate}

\section*{Como definir Procesos físicos}
\begin{enumerate}
\item Los siguientes son los procesos físicos mayores. 
  \begin{enumerate}
  \item electromagnetic
  \item hadronic
  \item transportation
  \item decay
  \item optical
  \item photolepton$\_$hadron
  \item parameterisation
  \end{enumerate}
\item Todos los procesos físicos se derivan de G4VProces.
\item El cual tiene metodos virtuales puros AtRestDoIt, AlongStepDoIt y
  PostStepDoIt. Los cuales basicamente determinan que se debe hacer en
  los diferentes estados que puede tener una partícula. Los anteriores
  métodos son invocados por el G4SteppingManager
\item Por cada uno de los anteriores existe otros tres metodos
  puramente virtuales GPIL
  \begin{enumerate}
  \item G4double PostStepGetPhysicalInteractionLength( const G4Track track, G4double
    previousStepSize, G4ForceCondition* condition )
  \item G4double AlongStepGetPhysicalInteractionLength( const G4Track
    track, G4double previousStepSize, G4double currentMinimumStep,
    G4double proposedSafety, G4GPILSelection* selection )
  \item G4double AtRestGetPhysicalInteractionLength( const G4Track
    track, G4ForceCondition* condition )
  \end{enumerate}
\item Adicionalmente a los metodos anteriores si se quiere definir un
  proceso físico, se deben implementar los siguientes metodos
  puramente virtuales
  \begin{enumerate}
  \item virtual G4bool IsApplicable(const G4ParticleDefinition)
  \item virtual void PreparePhysicsTable(const G4ParticleDefinition)
  \item virtual void BuildPhysicsTable(const G4ParticleDefinition)
  \item virtual void StartTracking()
  \item virtual void EndTracking()
  \end{enumerate}

\item Otros procesos físicos especializados podrian definirse a partir
  de las siguientes clases virtuales derivadas de G4VProcess
  \begin{enumerate}
  \item G4VRestProcess: Se usa unicamente el AtRestDoit (Captura
    Neutronica).
  \item G4VDiscreteProcess: usa unicamente PostStepDoIt (Compton Scattering)
  \item G4VContinuousDiscreteProcess: usa AlongStepDoIt y PostStepDoIt
    (Transporte, ioniascion.)
  \item G4VRestDiscreteProcess: usa AtRestDoIt y PostStepDoIt
    (aniquilación positrónica y decaimiento.)
  \item G4VRestContinuousProcess: usa AtRestDoIt y AlongStepDoIt
  \item G4VRestContinuousDiscreteProcess: usa AtRestDoIt,
    AlongStepDoIt y PostStepDoIt
  \end{enumerate}

\item G4ProcessManager contiene una lista de los procesos que una
  particula puede realizar y el orden en que se invocan los procesos. Hay
  un G4ProcessManager para cada partícula y este esta atado a la clase
  G4ParticleDefiniton (Cual es el tipo de relación?)
\item El ordenamiento de los procesos se registra con los metodos
  AddProcess() y SetProcessOrdering()
\item El G4ProcessManager puede desactivar y activar procesos con los
  metodos ActivateProcess() y InActivateProcess(), sin embargo esto se
  puede hacer unicamente despues de que se registre le proceso.
\item Registrar un proceso puede ser complicado ya que hay relaciones
  entre procesos. Para ello se inventaron el G4PhysicsListHelper. Por
  ejemplo:
  \begin{lstlisting}[language=C++, frame = single]
    void MyPhysicsList::ConstructProcess()
    {
      // Define transportation process
      AddTransportation();
      // electromagnetic processes
      ConstructEM();
    }
    
    void MyPhysicsList::ConstructEM()
    {
      // Get pointer to G4PhysicsListHelper
      G4PhysicsListHelper* ph = G4PhysicsListHelper::GetPhysicsListHelper();
      
      //  Get pointer to gamma
      G4ParticleDefinition* particle = G4Gamma::GammaDefinition(); 
      
      // Construct and register processes for gamma 
      ph->RegisterProcess(new G4PhotoElectricEffect(), particle);
      ph->RegisterProcess(new G4ComptonScattering(), particle);
      ph->RegisterProcess(new G4GammaConversion(), particle);
      ph->RegisterProcess(new G4RayleighScattering(), particle);
    }  
  \end{lstlisting}
\end{enumerate}
Unicamente los procesos pueden cambiar la información de los G4Track y
agregar tracks secundarios. Esto no se hace durante los DoIt, sino que
son un resultado de usar G4VParticleChange. La informacion del estado
final de un track incluyendo tracks secundarios es generada en los
Doit y almacenada en los G4VParticleChange. Los procesos físicos
actualizan la instancia de G4VParticleChange. Por lo tanto
G4VParticleChange es el responsable de actualizar el step. El stepping
manageer almacena los tracks secundarios.  G4Track se actualiaza al
finalizar todos los AlongStepDoIt y despues de cada PostStepDoIt.

G4VParticleChange, tiene los siguientes métodos virtuales puros.
\begin{enumerate}
\item virtual G4Step* UpdateStepForAtRest( G4Step* step),
\item virtual G4Step* UpdateStepForAlongStep( G4Step* step )
\item virtual G4Step* UpdateStepForPostStep( G4Step* step)
\end{enumerate}



\section*{Como generar eventos primarios}
\begin{enumerate}
\item Hay tres tipos de primary generators
  (G4VPrimaryGenerator). G4ParticleGun, G4GeneralParticleSource y G4HEPEvtInterface
\item Realmente el que esta generando el evento primario no es un
  objeto de G4VuserPrimaryGeneratorAction, el unicamente esta
  definiendo la forma como las particulas primarias son creadas. EL
  que realmente genera las particulas primarias es
  G4VPrimaryGenerator. Lo cual se hace cuando se llama al método
  puramente virtual de G4VuserPrimaryGeneratorAction
  generatePrimaries(), donde internamente un G4VPrimaryGenerator
  invoca al método generatePrimaryVertex()
\end{enumerate}

\section*{Sistema de Unidades}
\begin{enumerate}
\item Es obligatorio definirle unidades a los datos. Esto siempre se
  hace con la multiplicación 
\item Para sacar los datos en las unidades de interes basta con
  dividir el valor  $G4cout << KineticEnergy/keV $
\item Definir nuevas unidades
  \begin{lstlisting}[language=C++, frame = single]
    new G4UnitDefinition ( name, symbol, category, value )
    new G4UnitDefinition ( "km/hour" , "km/h", "Speed", km/(3600*s) );
  \end{lstlisting}
\end{enumerate}

\section*{Geant4 Conceptos Fundamentales}
Al igual que en matematicas una categoria es una estructura
algebraica, que consta de objetos conectados unos con otros con
flechas. Las flechas se pueden componer unas con otras de manera
asociativa, y para cada objeto existe una flecha que se comporta como
un elemento neutro. Por ejemplo en la categoria de conjuntos, los
objetos son los conjuntos y las flechas son las funciones
\begin{figure}
  \centering
  \includegraphics[width=\columnwidth]{im/classCategory.jpg}
  \caption[]{Esquema de la funcionalidad de Geant4 (daigrama de categorias)}
\end{figure}

La categoria global cubre el sistema de unidades, constantes y numeros
aleatorios.  Las categorias materialss y particles implementan las
propiedades físicas de partículas y materiales. El modulo geometry
define la geometria. La categoria track contiene los tracks y los
steps. Usados por la Categoria process la cual implementa modelos
fisicos. Estos procesos son llamados por la categoria tracking que
administra la evolucion de los tracks y provee de informacion de los
volumenes sensibles a los hits y digitalizacion.  La categoria Event
adminsitra los eventos en terminos de sus tracks.  La categoria Run
administra la recoleccion de eventos que comparten un haz comun e
implentacion de detectores. Finalmente la categoria readout administra
el pile-up.  La persistencia de objetos permite que los objetos vivan
despues de la terminacion de la aplicacion de un procesos y pueda ser
usado por otro proceso.

\begin{enumerate}
\item Hit: cuando ocurre un fenomeno fisico o un cambio de volumen (se
  llega a un punto de frontera) (de cierta forma esta definido en una
  region del espacio). Formalmente hablando es simplemente un exito,
  es decir algo ocurre.
\item Step: distancia(memoria) entre dos hits.
\item Track: toda la secuencia de steps, esta definido por particula.
\item Event: Una secuencia de tracks correlacionada (por ejemplo
  creacion de pares a tres track el foton inicial y el positron y
  electron creados )
\end{enumerate}

\section*{Geant4 Conceptos derivados}
\begin{enumerate}
\item Un G4PrimaryVertex almacena la informacion de posición y tiempo
  de las partículas primarias. Por lo tanto varias partículas
  primarias pueden estar en el mismo vertex.
\end{enumerate}

\section*{Run}
\begin{enumerate}
\item Un run es la unidad mas grande de simulacion.
\item Un run es una secuencia de eventos.
\item Durante un run la geometria, el montaje de detectores sensible y
  la fisica no pueden cambiar.
\item Pero el G4RunManager puede cambiar la geometría o la física al
  finalizar un run específico.
\item El \textbf{G4RunManager} manipula un loop de eventos atraves del método
  DoEventLoop(). Donde se construye un objeto G4Event y se asocian las
  particulas primarias, posteriormente el G4EventManager se encarga de
  asociar los hits y las trayectorias al evento.
\item Si los detectores sensibles y los modulos de digitalizacion son
  simulados entonces el run tiene acceso a G4VHitsCollection y
  G4VDigiCollection.
\item El runaction no crea el run solo modifica sus propiedades. El
  run es creado por el runmanager, por lo que si se quiere obtener el
  ID, lo cual es muy util en paralelizacion
\item El RunAction hace el papel del master y si no se define un Run
  interno tambien hace el papel de worker
\end{enumerate}

\section*{Event}
\begin{enumerate}
\item Los eventos contienen cuatro tipos de información.
  \begin{enumerate}
  \item Vertices primarios y particulas primarias.
  \item Trayectorias
  \item Colección de Hits (Solo si el detector es sensible)
  \item Colección de Digits 
  \end{enumerate}
\item El \textbf{G4EventManager} tiene las siguientes responsabilidades:
  \begin{enumerate}
  \item Convertir los G4PrimaryVertex y
    G4PrimaryParticle objetos, asociados con el evento actual en objetos
    G4Track. Todos los G4Track que representan particulas primarias se
    envían al \textbf{G4StackManager}
  \item Enviar los G4Track del G4StackManager hacia el
    G4TrackingManager. Los tracks unicamente se eliminan cuando ya han
    pasado por el G4TrackingManager.
  \item Cuando G4StackManager devuelve NULL, G4EventManager finaliza
    el evento actual que se esta procesando.
  \item Llama la metodos beginOfEventAction() y endOfEventAction()
    definidos en el G4UserEventAction
  \end{enumerate}
\end{enumerate}

\section*{Track}



\section*{Hits}
Un hit es una instantanea de una interaccion física de un track en la
region sensible de un detector. El hecho de que sea en una region
sensible es una de las principales diferencias respecto al
step. Básicamente en un hit se puede almacenar la siguiente
información.
\begin{enumerate}
\item Posición y tiempo del step.
\item Momento y Energía del track.
\item Energía depositada del step.
\item Información geométrica.
\end{enumerate}
Cuando se procesa un G4Event se producen muchos objetos G4VHit( esta
es la clase que representa el concepto de hit) los cuales se almacenan
atraves de una coleccion G4VHitsCollection, usualmente esta coleccion
de hits se instancia a traves de una clase template G4THitsCollection
la cual es herencia de G4VHitsCollection. El parametro del template es
el Hit que you haya definido.
\\
Una alternativa a G4THitsCollection es un G4THitsMap, el cual no
requiere un G4VHit, pero en su lugar requiere de una variable que
pueda mapearse como una llave tipo entero. Tipicamente la llave es el
número de copia del volumen. Los mapas son muy utiles cuando no se
requiera analizar los datos evento por evento, sino simplemente
acumularlos. G4THitsMap es una herencia de G4VHitsCollection.
\\
Los objetos de G4THitsMap y G4THitsCollection se almacenan en
G4HCofThisEvent el final del evento.

\subsection*{G4VSensitiveDetector}
Esta clase representa el concepto de detector sensible. Este concepto
es posterior en Geant y se creo con el fin de dotar a las simulaciones
con la misma estructura de los experimentos.  Antes de eso se
utilizaban los scores de los hooks, es decir basicamente se almacenaba
la informacion unicamente de los volumenes logicos que se les ponga el
nombre de "scorer" pero al final de cuentas en cada hook se preguntaba
si se esta dentro del scorer o por fuera.
\\
La primera obligacion de los detectores sensibles es contruir los hits
usando la información de los steps a lo largo del track de una
partícula. Esto se hace a traves del método ProcessHits, el cual usa
un G4Step como entrada o un G4TouchableHistory en caso de que las
geometrias sean de lectura (Readout)
\\
La segunda obligacion es construir el detector usando el método
G4VUserDetectorConstruction::ConstructSDandField(), en esta parte se
asigna uno o mas detectores sensibles a uno o varios G4LogicalVolume
$ SetSensitiveDetector("LogicalV Name", mySensitiveDetector ,
true);$. Adicionalmente los punteros de estos volumenes logicos, deben
registrarse a G4SDManager. El G4SDManager es un singleton
\\
Hay tres metodos virtuales mayores:
\begin{enumerate}
\item ProcessHits(G4Step* , G4TouchableHistory*): Este metodo es llamado por el G4SteppingManager,
  cuando un se crea un step in el G4LogicalVolume que es el apuntador
  que se le asigno al detector sensible.
\item Initialize(G4HCofThisEvent ): Se llama al inicion de cada
  evento. Tambie se puede asociar el hits collection para durante el
  proceso de los eventos se realize la digitalizacion.
\item EndOfEvent(G4HCofThisEvent ): se llama al final del event, se
  usa ocasionalmete para asociar los G4HCofThisEvent del detector
  sensible con los objetos argumento.
\end{enumerate}

Es posible asignar multiples detectores sensibles a un mismo volumen
lógico, esto se logra a traves de G4MultiSensitiveDetector

\subsection*{Accediendo a las colecciones de hits }
A las colecciones de hits es importante acceder en casos como:
\begin{enumerate}
\item Digitalización
\item Filtrado de Eventos en G4VUserStackingAction
\item El final del evento para analisis simple.
\item Dibujar y pintar los hits.
\end{enumerate}

\begin{lstlisting}[language=C++, frame = single]
     G4SDManager* fSDM = G4SDManager::GetSDMpointer();
  G4RunManager* fRM = G4RunManager::GetRunManager();
  G4int collectionID = fSDM->GetCollectionID("collection_name");
  const G4Event* currentEvent = fRM->GetCurrentEvent();
  G4HCofThisEvent* HCofEvent = currentEvent->GetHCofThisEvent();
  MyHitsCollection* myCollection = (MyHitsCollection*)(HC0fEvent->GetHC(collectionID));
\end{lstlisting}


\subsection*{G4MultiFunctionalDetector y G4VPrimitiveScorer}
G4MultiFunctionalDetector es una clase específica derivada
G4VSensitiveDetector tambia a los volumenes lógicos de interés se les
asigna un detector sensible.
$ SetSensitiveDetector("LogicalV Name", myMultiFunctionalDetector ,
true);$
\\
Un G4VPrimitiveScorer no debería ser comportido por mas de un
G4MultiFunctionalDetector. G4VPrimitiveScorer crea un G4THitsMap por
evento. El mapa tiene el mismo nombre que el primitivescorer. Todos
los primitivescorer generan mapas de doubles G4THitsMap<G4double>

\subsection*{Digitalizacion}
G4VDigitizerModule se encarga tipicamente de simular :
\begin{enumerate}
\item ADC
\item readout scheme
\item datos crudos.
\item logica de trigers
\item pile up
\end{enumerate}
G4VDigi es una clase abstracta que representa un digito.
G4TDigiCollection is a template class for digits collections, which is
derived from the abstract base class G4VDigiCollection. G4Event has a
G4DCofThisEvent object, which is a container class of collec- tions of
digits. The usages of G4VDigi and G4TDigiCollection are almost the
same as G4VHit and G4THitsCollection.
\\
El trabajo interesante lo hace G4VDigitizerModule a traves de un
metodo puramente virtual, Digitize(). Osea el usuario siempre lo debe
definir.
\\
G4DigiManager es un singleton que administra los modulos de
digitalizacion. Todos los modulos de digitalizacion deben registrarse
fDM->AddNewModule(myDM);
\begin{lstlisting}[language=C++, frame = single]
G4DigiManager * fDM = G4DigiManager::GetDMpointer();
G4int myHitsCollID = fDM->GetHitsCollectionID( "hits_collection_name" );
G4int myDigiCollID = fDM->GetDigiCollectionID( "digi_collection_name" );
MyHitsCollection * HC = fDM->GetHitsCollection( myHitsCollID );
MyDigiCollection * DC = fDM->GetDigiCollection( myDigiCollID );
MyHitsCollection * HC = fDM->GetHitsCollection( myHitsCollID, n );
MyDigiCollection * DC = fDM->GetDigiCollection( myDigiCollID, n );
\end{lstlisting}



\subsection*{Accediendo a la colección de dígitos}
G4DigiManager al igual que G4SDManager tiene acceso a los hits, pero
adicionalmente G4DigiManager tiene acceso a los digitos. Una
G4Trajectory es creada pro G4TrackingManager cuando un G4Track es
pasdo por el G4EventManager.

\subsection*{Tracking y Physics }
G4Trajectory y G4TrajectoryPoint son clases heredadas de G4VTrajectory
y G4VTrajectoryPoint.

Una G4Trajectory tiene los siguientes miembros:
\begin{enumerate}
\item ID del track y su padre
\item Nombre de la particula, carga y codigo PDG.
\item Una coleccion de G4TrajectoryPoint
\end{enumerate}

\begin{enumerate}
\item G4TrackingManager: comunica las categorias event, track y
  tracking. El recibe un track del event manager y finaliza la
  trackializacion. Este manager agrega los punteros G4Trajectory y
  G4UserTrackingAction al G4SteppingManager
\item G4SteppingManager: Juega un papel esencial en la trackialization
  de la particula.
\end{enumerate}

Algoritmo para manipular los steps.
\begin{enumerate}
\item Inicio del Step.
\item Si la particula se detiene los procesos atRess definen una
  longitud temporal para el paso. Se llaman primeros los procesos que
  la menor duracion (length in time)
\item Todos los procesos continuos o discrtes proponen una longitd del
  paso (real, espacial ). Se toma la longitud mas pequeña.
\item El geometry navigator calcula la distancia minima a una frontera
  del volumen si el paso del proceso es menor, entonces se selecciona
  la longitud del paso del proceso. En caso contrario se calcula la
  distancia a la siguiente frontera mas cercana. Siempre se toma el
  minino entre ambas distancias.
\item Se actualiza las propiedades físicas de todos los procesos
  continuo como la energia cinetica al final es la suma de las
  energias cineticas, la posicion y el tiempo. Y se almacena la
  SecundaryList
\item Se mira la energía cinetica para ver si termino los procesos
  continuos.
\item Se invocan a los procesos discretos. Se actualizan las
  propiedades físicas y se guarda la SecundaryList.
\item Se recalcula "Safety"
\item Si el step es limitado por el volumen, empujar la partícula al siguiente volumen.
\item Manipular la informacion del hit.
\item Llamar la intervencion del usuario G4UserSteppingAction
\item Guardar datos de la trayectoria.
\item Actualizar el camino libre medio de procesos discreto.
\item Redefinir la longitud maxima de interaccion de los procesos
  discretos, si la particula padre esta aun viva.
\item Fin del Step.
\end{enumerate}

\section*{Stacking}
G4StackManager has three stacks, named urgent, waiting and
postpone-to-next-event, which are objects of the G4TrackStack
class. By default, all G4Track objects are stored in the urgent stack
and handled in a "last in first out" manner. In this case, the other
two stacks are not used. However, tracks may be routed to the other
two stacks by the user-defined G4UserStackingAction concrete class.

If the methods of G4UserStackingAction have been overridden by the
user, the postpone-to-next-event and waiting stacks may contain
tracks. At the beginning of an event, G4StackManager checks to see if
any tracks left over from the previous event are stored in the
postpone-to-next-event stack. If so, it attemps to move them to the
urgent stack. But first the PrepareNewEvent() method of
G4UserStackingAction is called. Here tracks may be re-classified by
the user and sent to the urgent or waiting stacks, or deferred again
to the postpone-to-next-event stack. As the event is processed
G4StackManager pops tracks from the urgent stack until it is empty. At
this point the NewStage() method of G4UserStackingAction is called. In
this method tracks from the waiting stack may be sent to the urgent
stack, retained in the waiting stack or postponed to the next event.

\section*{Tecnicas de Preferenciación de eventos (Event Biasing)}
El objetivo es reducir el tiempo de simulación escogiendo aquellos
eventos que son relevantes para los intereses de la simulación. Por
ejemplo para calcular el blindaje, Algunas tecnicas de preferenciacion
de eventos son:
\begin{enumerate}
\item Production cuts y threshold
\item Basados en la Geometria (Importance sampling para una region o
  volumen) o un Weight window.
\item Basados por particulas (Tomar solo las particulas secundarias
  mas energéticas. No se conserva la energía)
\item Basados en eventos primarios ( escoger el tipo de evento mas
  importante, el momento la distribucion. En este caso hay que tener
  un solido conocimiento de la fíßica requerida)
\item Incrementar la sección eficaz de un proceso
\item G4VSDFilter.??
\end{enumerate}

\section*{Geant4 Notas apartes}
\begin{enumerate}
\item La geometria tambien se puede definir en autocad y luego
  importarla a Geant. aplicara lo mismo para SolidWorks?
\item El usuario no debería eliminar la geometría manualmente, de eso
  se encarga VolumeStorer, administrado por el RunManager.
\item El unico volumen que no debe ponerse de un volumen madre es el
  mundo, en su lugar se pone un 0, de hecho esto es lo que permite
  distingui al mundo de los otros volumenes.
\item A todos los solidos se les puede sacar el volumen y el área
  superficial a traves de los métodos. GetCubicVolume() y
  GetSurfaceArea(). Para solidos complejos utiliza integración por
  montecarlo.
\item El mundo tampoco se puede rotar.
\item Las rotaciones se definen respecto al volumen madre.
\item Para todo tipo de partícula existe una librería granular
  correspondiente.
\item Los G4Isotope, G4Element y G4Material no deberían eliminarse.
\item Los primary generators como G4ParticleGun se deben eliminar en
  el destructor de G4VUserPrimaryGeneratorAction
\item Se puede invocar mas de un generator o el mismo mas de una ves
  para producir eventos primarios mucho mas complicados.
\item Uno puede crear su propio RunManager como herencia de
  G4RunManager.
\item EL sobrelapamiento de eventos se puede lograr de diferentes
  modos, el primero usando multiples generadores (G4VPrimaryGenerator)
  o simplemente mezclando hits o digitos.
\item Los G4MultiFunctionalDetector tienen Biasing scorers específicos.
\item Todos los G4VPrimitiveScorer deben unsar un G4THitsMap.
\item Los messenger deben tener precaucion de si la que se va a
  cambiar se hace antes o despues de correr el run.
\item En Geant4 se pueden crear mundos paralelos. Es decir pueden
  haber varios mundos en la simualación.
\item Los tracks de las particulas negativas se pintan de rojo, las
  neutrales verde y las positivas azules.
\item While in sequential mode the action classes are instatiated just
  once, via invoking the method: B1ActionInitialization::Build() in
  multi-threading mode the same method is invoked for each thread
  worker and so all user action classes are defined thread-local
  A run action class is instantiated both thread-local and global
  that's why its instance is created also in the method
  B1ActionInitialization::BuildForMaster() which is invoked only in
  multi-threading mode.
\item G4Accumulable<> type instead of G4double and G4int types is used
  for the B3aRunAction data members in order to facilitate merging of
  the values accumulated on workers to the master. Currently the
  accumulables have to be registered to G4AccumulablesManager and
  G4AccumulablesManager::Merge() has to be called from the users code.
\item Para visualizar con OpenInventor a que tirarle un nucleo
  adicional.
\end{enumerate}





\section*{Acceder a información de la simulación}
Para acceder a la informacion se pueden acceder directamente a cada
una de las partes esenciales de la estructura de las simulaciones de
geant4. A traves de los User hook (Stepping, Event, Trackingaction,
Runaction). Otra forma es usar Multifuctional detector mas scoring, de
esta forma el runmanager genera la estructura de la simulacion, pero
se requiere acceder a colleccions de hit atraves de un event. Y esto
obliga a la generación de mapas.  La otra forma es crear una herencia
de Vhits que le permita obtener la informacion de relevacia de los
hits.En los hits se puede almacenar la informacion asociada con los
steps. Energia depositada, tiempo y posicion por step, momento y
energia cinetica del track, información geométrica. Luego se hace una
coleccion de hits.




\section*{Interpretación de algunas lineas de código}
\begin{enumerate}
\item
  \begin{lstlisting}[language=C++, frame = single]
    B1RunAction::B1RunAction()
    : G4UserRunAction(),
    fEdep(0.),
    fEdep2(0.)
  \end{lstlisting}
 

  Significa que el constructor esta siendo heredado y que unas variables
  se estan inicializando, en el fondo lo relevante del constructor esta
  siendo heredado, la variables que se inicializan son las que conocemos
  como datos miembro de la clase.
\end{enumerate}


\section*{Extensiones}
\textbf{Definición} una macro es un fragmento de codigo, donde se use
el nombre del macro se reemplaza por el contenido del macro. Hay de
dos tipos objetos y funciones
\begin{enumerate}
\item .in son para ejecutar en batch mode
\item .mac fragmento de cogido para ejecutar en batch mode no difiere
  del .in
\end{enumerate}

\section*{Trucos Generales}
\begin{enumerate}
\item Para visualizar un arbol de root se usa el comando: TBrowser p
\item Las librerias que no empienzan con el nombre de ejemplo o del
  codigo B1, entonces son librerias propias de Geant
\item Si no se ha guardado un archivo src o include entonces cmake
  genera problemas
\item /usr/share/Geant4-10.2.1/examples/ Aca estan los ejemplos
  actualizados de Geant4
\item Para cambiar los nombres de los archivos debe mirar y
  modificarse primero la CMakeList.txt y finalmente el GNUmakefile
\end{enumerate}

\section*{Algunos Comandos usados en problemas}
\begin{enumerate}
\item grep -re palabras /dir
\item Importante para lograr esto es importante borrar el cache
  (CMakeCache.txt) antes de ejecutar el comando. Como no existe cmake
  clean toca borrar todos los archivos generados por el CMake para
  "curarse en salud" o simplemente eliminar todo lo que contenga la
  carpeta build $rm -rf *$
  \begin{lstlisting}
    cmake -D WITH_GEANT4_UIVIS=OFF ../B1/
    cmake -DG4MULTITHREADED=OFF ../B1/
  \end{lstlisting}
\item gcc --version
\end{enumerate}

\section*{Algunas herramientas importantísimas de aprender}
\begin{enumerate}
\item ULM (Unified Modeling Language): Es un lenguaje gráfico que
  permite usar notacion industrial estandar para el desarrollo de
  aplicaciones.
\item Test-driven development (TDD): Muy útil para hacer pruevas de
  validación y trabajar en grupos de que la estructura del programa
  sea la que se acordó
\end{enumerate}

\section*{Cosas que me salté}
\begin{enumerate}
\item GPS
\item CLHEP library
\item Biasing Events.
\item Revision detalla de Geometry y Electromagnetic Field
\item Persintency
\item Geometrias Paralelas
\end{enumerate}


\end{document}
